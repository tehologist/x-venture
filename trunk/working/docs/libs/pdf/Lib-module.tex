%
% API Documentation for x-venture
% Package Lib
%
% Generated by epydoc 3.0alpha3
% [Thu Oct 19 18:52:09 2006]
%

%%%%%%%%%%%%%%%%%%%%%%%%%%%%%%%%%%%%%%%%%%%%%%%%%%%%%%%%%%%%%%%%%%%%%%%%%%%
%%                          Module Description                           %%
%%%%%%%%%%%%%%%%%%%%%%%%%%%%%%%%%%%%%%%%%%%%%%%%%%%%%%%%%%%%%%%%%%%%%%%%%%%

    \index{Lib \textit{(package)}|(}
\section{Package Lib}

    \label{Lib}
Various classes for use in implementing generic mud.


%%%%%%%%%%%%%%%%%%%%%%%%%%%%%%%%%%%%%%%%%%%%%%%%%%%%%%%%%%%%%%%%%%%%%%%%%%%
%%                                Modules                                %%
%%%%%%%%%%%%%%%%%%%%%%%%%%%%%%%%%%%%%%%%%%%%%%%%%%%%%%%%%%%%%%%%%%%%%%%%%%%

\subsection{Modules}

\begin{itemize}
\setlength{\parskip}{0ex}
\item \textbf{env}: These are helper functions for implementing a basic mud.



  \textit{(Section \ref{Lib:env}, p.~\pageref{Lib:env})}

\item \textbf{mudParser}: This is a simple parser for a mud grammar.



  \textit{(Section \ref{Lib:mudParser}, p.~\pageref{Lib:mudParser})}

\item \textbf{simple\_client}: An example of the simplest possible client.



  \textit{(Section \ref{Lib:simple_client}, p.~\pageref{Lib:simple_client})}

\end{itemize}

    \index{Lib \textit{(package)}|)}
